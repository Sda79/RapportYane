\documentclass{article}


\begin{document}
	\title{Yet Another Network Simulator Développement d'un simulateur/émulateur de réseaux}
	\date{2018-18-09}
	\author{Sylvain Daste\\Avec l'aide de Mr. Chaput}
	
	\maketitle
  	\newpage
	\tableofcontents
  	\newpage
	\section{Remerciement}
Je tiens à remercier mon maitre de stage, MrEmmanuel Chaput, pour son accueil, le temps passé ensemble et toute l'aide qu'il m'a apporté. Grâce aussi à sa confiance j'ai pu m'accomplir totalement dans mes missions. Il fut d'une aide précieuse dans les moments les plus délicats. Je remercie également l'Irit de m'avoir chaleuresement acceuilli au sein de leur équipe.

	\newpage
	\section{Introduction}
L'objectif de ce stage était de fournir un logiciel simple capable de simuler des réseaux internet. Mr Chaput avais déjà fournit un socle de ce logiciel avec les fonctionnalitées les plus basiques. Mes missions étaient de :
\begin{itemize}
\item fournir une documention des principales fonctionnalitées à la fois d'un point de vue utilisateur et developpeur.
\item développer de nouvelles fonctionnalité permettant au programme d'étendre ses capacités d'émulation
\item faciliter le developpement de fonctionnalité developper dans le futur
\end{itemize}
Tout le developpement de ce logiciel devait être réalisé grâce à un code propre et compréhensible par tous. 
	\newpage	
	\section{Yet Another Network Emulator}
		\subsection{Fonctionnement de Yane}
		\subsection{Objectifs de Yane}
		\subsection{Technologies}
		\newpage
	\section{Conclusion}
\end{document}