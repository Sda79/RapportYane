\documentclass{article}


\begin{document}
	\title{Yet Another Network Simulator\\Développement d'un simulateur de réseaux}
	\date{Toulouse 2018-18-09}
	\author{Sylvain Daste\\Avec l'aide de Mr. Chaput Enseignant Chercheur à l'Irit\\ENSEEIHT\\IRIT}
	\maketitle



  	\newpage
	\tableofcontents
  	\newpage
	
	
	\section{Remerciement}
\paragraph{}
Je tiens à remercier mon maitre de stage, MrEmmanuel Chaput, pour son accueil, le temps passé ensemble et toute l'aide qu'il m'a apporté. Grâce aussi à sa confiance j'ai pu m'accomplir totalement dans mes missions. Il fut d'une aide précieuse dans les moments les plus délicats. Je remercie également l'Irit de m'avoir chaleuresement acceuilli au sein de leur équipe.
	\newpage
	
	
	
	
	
	\section{Introduction}
	\subsection{Objectifs du stage}
\paragraph{}
L'objectif de ce stage était de fournir un logiciel simple capable de simuler/émuler des réseaux internet. Mr Chaput avais déjà fournit un socle de ce logiciel avec les fonctionnalités les plus basiques. Mes missions étaient de :
		\begin{itemize}
			\item fournir une documention des principales fonctionnalités à la fois d'un point de vue utilisateur et développeur.
			\item développer de nouvelles fonctionnalités permettant au programme d'étendre ses capacités d'émulation
			\item permettre l'émulation du réseaux grâce à différentes technologies notamment Docker.
			\item développer Yane de façon à faciliter l'intégration de nouvelles fonctionnalités.
			\item essayer plusieurs interface d'intéraction avec Yane (tmux, terminator,... )
		\end{itemize}
Le dévoloppement et le suivit de Yane est accessible sur Github : https://github.com/Manu-31/yane.git
	\subsection{Objectifs de Yane}
	\paragraph{}

	Yane a pour objectif d'émuler des réseaux complets. Les réseaux simulés par Yane sont des réseaux internet. Par conséquent Yane se base uniquement sur IP. On doit être en mesure de simuler jusqu'à 20 machines environ. Yane doit également pouvoir simuler différent type de machines. Par exemple émuler un serveur DNS ou des routeurs. Une fois le réseau émulé l'utilisateur doit pouvoir intéragir avec celui-ci. Les intéractions provoqueront l'envoie de paquets entre les différentes machines. 
	
	\paragraph{}
	Mr. Chaput avait déjà développé une base très simple de Yane. Il avait également fournit un ensemble d'exemples d'utilisation de Yane. Nous allons passer en revu les principaux exemples car ils permettent de démontrer les capacités que Yane devra acquérir.

	\newpage	



	\section{Yet Another Network Emulator}
		\subsection{Fonctionnement de Yane}
		\subsection{Objectifs de Yane}
		\subsection{Technologies}
		\newpage
	\section{Conclusion}
\end{document}