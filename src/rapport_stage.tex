\documentclass{article}
\usepackage[utf8]{inputenc}
\usepackage[cyr]{aeguill}
\usepackage[francais]{babel}
\usepackage{listings}

\begin{document}
	\title{Yet Another Network Simulator\\Développement d'un outil de simulation de réseaux}
	\date{Toulouse 2018-18-09}
	\author{Sylvain Daste\\Avec l'aide de Mr. Chaput Enseignant Chercheur à l'Irit\\ENSEEIHT\\IRIT}
	\maketitle



  	\newpage
	\tableofcontents
  	\newpage
	
	
	\section{Remerciement}
	\paragraph{}
	Je tiens à remercier mon maitre de stage, Mr. Emmanuel Chaput, pour son accueil, le temps passé ensemble et toute l'aide qu'il m'a apporté. Grâce aussi à sa confiance j'ai pu m'accomplir totalement dans mes missions. Il fut d'une aide précieuse dans les moments les plus délicats. Je remercie également l'Irit de m'avoir chaleuresement acceuilli au sein de leur équipe.
	\newpage


	\section{Introduction}


	\subsection{Objectifs de Yane}
	
	\paragraph{}
	Yet Another Network Simulator a pour objectif d'émuler des réseaux complets. Les réseaux simulés par Yane sont des réseaux internet. Par conséquent Yane se base uniquement sur les protocoles TCP/IP (IPv4 seulement). Il doit être en mesure d'émuler ou de virtualiser jusqu'à 20 stations. Les stations peuvent être de nature différentes. Par exemple émuler un serveur DNS, des routeurs, des bases de données. Ces différents besoins imposent à Yane de pouvoir émuler la plusieurs OS ou de virtualiser différentes applications. Yane permettra ainsi de simuler des réseaux de taille moyenne sur une seule machine et ainsi être utilisé en TP à l'ENSEEIHT afin d'apprendre à manipuler et comprendre les réseaux.
	
	\paragraph{}
	Le choix fait par Mr. Chaput était de rendre Yane modulaire afin de pouvoir intégrer différentes techniques d'émulation au fur et à mesure de son développement.

	\paragraph{}
	Une fois le réseau émulé, l'utilisateur doit pouvoir intéragir avec celui-ci. Les intéractions provoqueront l'envoie de paquets entre les différentes stations. L'utilisateur aura aussi la possibilité d'écouter le réseau.
	
	\paragraph{}
	Mr. Chaput avait déjà développé une base simple de Yane. Il avait fournit un ensemble d'exemples d'utilisation de Yane. Nous passerons en revu les principaux exemples car ils permettent de comprendre le fonctionnement de Yane.
	
	
	\subsection{Objectifs du stage}

	\paragraph{}
	L'objectif de ce stage était d'améliorer Yane, un logiciel simple capable de simuler/émuler des réseaux internet. Mr Chaput avais déjà fournit un socle avec les fonctionnalités les plus basiques. Mes missions étaient de :
			
			\begin{itemize}
				\item fournir une documention des fonctionnalités à la fois d'un point de vue utilisateur que développeur.
				\item développer de nouvelles techniques d'émulation, Docker notamment.
				\item développer Yane de façon à faciliter l'intégration de nouvelles fonctionnalités.
				\item essayer plusieurs interfaces d'intéraction avec Yane (tmux, terminator,...)
			\end{itemize}
	
		
	\paragraph{}
	Le dévoloppement de Yane est accessible sur Github :
	\newline https://github.com/Manu-31/yane.git ce qui permet un suivit agile.
	\paragraph{Annonce du plan}

	\newpage	



	\section{Yane - Utilisation}
		Dans cette partie nous allons rapidement voir comment Yane peut être utilisé et les fonctionnalités qu'il propose. Puis nous verrons quelques exemples d'utilisation.		

		\subsection{Simuler un réseau}
		Une fois que l'utilisateur a spécifier le réseau à simuler, il doit créer un fichier yane.yml. Ce fichier sera la représentation du réseau. Mais l'utilisateur pourra aussi, via ce fichier, demander plusieurs chose à Yane, grâce aux différentes fonctionnalités qu'il possède. Dont par exemple, spécifier quelles méthodes de virtualisation ou d'émulation doivent être utilisé.
		
		\paragraph{}
		Le code suivant montre un exemple simple de ce que j'ai dû dans un premier temps réaliser :
		
		\begin{lstlisting}[basicstyle=\small, caption=Réseau à 2 stations]
network:
name: basic
version: 1.0
hosts:
	- host-a
	- host-b

links:
	- host-a:v0:192.168.1.1/24!host-b:v0:192.168.1.2/24

consoles:
	- all

dumpif:
- all
		\end{lstlisting}

		\paragraph{}
		Yaml permet d'être assez explicite. On peut clairement lire que ce fichier génerera un réseau avec 2 stations reliées par un lien. On peut spécifier la configuration des interfaces des stations grâce à la balise `link`. Si l'on souhaite relier plus de 2 machines entre elles Yane est capable d'émuler un pont. Il faut alors employer la méthode suivante :

		\begin{lstlisting}[basicstyle=\small, caption=Les ponts]
			...
bridges :
    - __BRIDGE_NAME__
      interfaces: host-a:v0:192.168.1.1/24!\
		host-b:v0:192.168.1.2/24!\
		host-c:v0:192.168.1.3/24
		\end{lstlisting}
		
		
		\subsection{Exemples d'utilisation}
		La première mission de mon stage était de fournir une documentation de Yane mais également de comprendre Yane en fournissant de nouveaux exemples d'utilisation de Yane.
		
		\newpage
	
	
	
	
	
		\section{Yane - Fonctionnement et architecture}
		\subsection{Choix des languages}
			\subsubsection{Développeur}
			Yane a été développé en BASH. Celui-ci permet de contrôler les "namespaces" et Docker facilement. Par conséquent cela à permis un développe\-ment plus rapide de Yane. Cependant BASH a montré ses limites notamment lorsque Yane à besoin de faire les choses par soi-même. Par exemple de simples calculs en BASH peuvent être délicats à mettre en oeuvre.

			\subsubsection{Utilisateur}
			L'utlisateur devra fournir une description des réseaux en Yaml. Yet Another Markup Langage est un language de représentation de données. Pour plus d'informations sur Yaml: http://yaml.org
		\subsection{Modularité}
		\subsection{YAML}
		\subsection{Tmux}
		\newpage
	\section{Docker}
		\subsection{Docker | Bases}
		\subsection{Docker et Yane}
		\subsection{}
		\newpage
	\section{L'intéraction avec Yane}
		\subsection{tmux}
		\newpage
	\section{Conclusion}
\end{document}